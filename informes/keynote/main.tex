\documentclass[spanish]{beamer}

\include{conf/preconfig}
\include{conf/packages}
\include{conf/config}
\include{beamerconf}
%
%
\usetheme{Bergen}
\usecolortheme{orchid}
%
%
%
\title{Identificación de edificios bla bla}
\author{Esteban C. Fornal \and Christian N. Pfarher \and Mauro J. Torrez}
\date{\today}

\begin{document}
%
\frame{\titlepage}

\section[Outline]{Introducción}
\begin{frame}{Introducción}
%\frametitle{Introducción}
\begin{itemize}
\item<1-> Objetivo
\begin{itemize}
\item Implementar un detector de edificios bla bla
\end{itemize}
\end{itemize}
\end{frame}

\subsection[Outline]{Control borroso}
\begin{frame}{¿Porqué usamos controladores borrosos?}
\begin{itemize}
\item<1-> Facilidad en la especificación de las reglas (lenguaje natural)
\item<2-> Eficiencia
\end{itemize}
\end{frame}

\begin{frame}{Sensores}
%% \begin{align*}
%% \alpha_0=0.9e^{j\frac{\pi}{4}},\qquad\alpha_1=0.9e^{j\frac{3\pi}{4}}
%% \end{align*}
%\includegraphics[width=9cm]{img/sensores}
\end{frame}

\begin{frame}{Entradas}
  \begin{itemize}
  \item Distancia medida por el sensor frontal izquierdo
  \item Distancia medida por el sensor frontal derecho
  \item Distancia medida por el sensor lateral izquierdo
  \item Distancia medida por el sensor lateral derecho
  \item Velocidad actual de coche
  \end{itemize}
\end{frame}
\begin{frame}{Salidas}
  \begin{itemize}
  \item Dirección %$\in [-1,1]$
  \item Presión acelerador %$\in [0,1]$
  \item Presión freno% $\in [0,1]$
  \end{itemize}
\end{frame}


\begin{frame}{Diagrama}
  %\includegraphics[width=9cm]{img/diag-modulos}
\end{frame}

\begin{frame}{Fuzzificación de las entradas}
  \begin{itemize}
  \item Codificación correlación-mínimo
  \end{itemize}
%\includegraphics[width=8cm]{img/conj-entradas}
\end{frame}

\begin{frame}{Defuzzificación de las salidas}
  \begin{itemize}
  \item La salida la obtenemos mediante el método de los centroides:
    \begin{equation*}
      y=\frac{\sum_{i=1}^{m}{a_ic_i}}{\sum_{i=1}^{m}{a_i}}
    \end{equation*}
  \end{itemize}
  %\includegraphics[width=8cm]{img/conj-salidas}
\end{frame}

\begin{frame}{Evaluación}
\framesubtitle{Parámetros a evaluar}
  \begin{itemize}
  \item Cantidad de colisiones
  \item Distancia total recorrida
  \item Tiempo fuera de la pista
  \end{itemize}
\end{frame}

\begin{frame}{Evaluación}
\framesubtitle{Características}
  \begin{itemize}
  \item Se evaluaron 2 conjuntos de reglas borrosas y jugadores humanos
  \item 2 cantidades de competidores
  \item Distintos comportamientos de los competidores
  \item Simulaciones de 3 minutos cada una
  \item 3 simulaciones por configuración
  \end{itemize}
\end{frame}

\begin{frame}{Resultados}
%\framesubtitle{\large{Cantidad de colisiones}}
Cantidad de colisiones

  %\includegraphics[width=9cm]{img/nroColisiones}
\end{frame}



\begin{frame}{Resultados}
%\framesubtitle{\large{Cantidad de colisiones}}
Distancia recorrida

  %%\includegraphics[width=9cm]{img/distanciaRecorrida}
\end{frame}

\begin{frame}{Resultados}
%\framesubtitle{\large{Cantidad de colisiones}}
Tiempo fuera de la pista

  %%\includegraphics[width=9cm]{img/tiempoFuera}
\end{frame}

\begin{frame}{Conclusiones}
  \begin{itemize}
  \item Sensores
    \begin{itemize}
    \item Cantidad
    \item Tipo
    \item Configuración
    \end{itemize}
  \item Resultados satisfactorios para la información disponible
  \end{itemize}
\end{frame}

\end{document}








