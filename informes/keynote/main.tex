\documentclass[spanish]{beamer}

\include{conf/preconfig}
\include{conf/packages}
\include{conf/config}
\include{beamerconf}
%
%
\usetheme{Bergen}
\usecolortheme{orchid}
%
%
%
\title{Identificación de edificios y monumentos a partir de fotografías tomadas 
con dispositivos móviles}
\author{Esteban C. Fornal \and Christian N. Pfarher \and Mauro J. Torrez}
\date{\today}

\begin{document}
%
\frame{\titlepage}

\section[Outline]{Introducción}
\begin{frame}{Introducción}
%\frametitle{Introducción}
\begin{itemize}
\item<1-> Objetivo
\begin{itemize}
\item  Identificar edificios y monumentos a partir de fotografías
\end{itemize}
\end{itemize}
\end{frame}

\subsection[Outline]{Herramientas}
\begin{frame}{Dos técnicas:}
\begin{itemize}
\item<1-> Transformada de Hough
\item<2-> Histograma
\end{itemize}
\end{frame}

\begin{frame}{Transformada de Hough}
%% \begin{align*}
%% \alpha_0=0.9e^{j\frac{\pi}{4}},\qquad\alpha_1=0.9e^{j\frac{3\pi}{4}}
%% \end{align*}
%\includegraphics[width=9cm]{img/sensores}
\end{frame}


\begin{frame}{Histograma}
%% \begin{align*}
%% \alpha_0=0.9e^{j\frac{\pi}{4}},\qquad\alpha_1=0.9e^{j\frac{3\pi}{4}}
%% \end{align*}
%\includegraphics[width=9cm]{img/sensores}
\end{frame}


\begin{frame}{Método}
  \begin{itemize}
  \item ambas tecnicas !!!bla
  \end{itemize}
\end{frame}

\begin{frame}{Pruebas}
  \begin{itemize}
  \item Dirección %$\in [-1,1]$
  \item Presión acelerador %$\in [0,1]$
  \item Presión freno% $\in [0,1]$
  \end{itemize}
\end{frame}


\begin{frame}{Resultados}
  %\includegraphics[width=9cm]{img/diag-modulos}
\end{frame}

\begin{frame}{Conclusiones}
  \begin{itemize}
  \item Sensores
    \begin{itemize}
    \item Cantidad
    \item Tipo
    \item Configuración
    \end{itemize}
  \item Resultados satisfactorios para la información disponible
  \end{itemize}
\end{frame}

\end{document}








