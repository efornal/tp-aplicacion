
%Estilos de texto
\newcommand{\resalt}{\colorbox{green}}	%Resaltado - Fondo verde
\newcommand{\sfbf}{\bfseries\textsf}	%Slanted + Bold
\newcommand{\eng}{\textit}			%Palabra en inglés - Itálica
\newcommand{\mean}{\textsl}			%Significado de una sigla - Slanted
\newcommand{\defin}{\textbf}			%Definición - Negrita
\newcommand{\R}{\mathds}			%Para escribir R de Reales, N de Nat
\newcommand{\N}{\mathbf}			%Para escribir R de Reales, N de Nat
\newcommand{\lil}[1]{\footnotesize #1}
\newcommand{\mono}[1]{{\tt #1}}         % Monoespaciado


%Símbolos
\newcommand{\y}{\wedge}			%Y (Lógica)
\newcommand{\ve}{\vee}			%O (Lógica)
\newcommand{\ent}{\supset}		%Entonces (Lógica)
\newcommand{\dimp}{\leftrightarrow}	%Doble implicativo, equivalencia (Lógica)
\newcommand{\sii}{\leftrightarrow}	%Si y sólo si (Lógica)
\newcommand{\equi}{\equiv}		%Equivalencia (Lógica)
\newcommand{\portanto}{\vdash}	%Por lo tanto (Lógica)
\newcommand{\por}{\cdot}		%Producto punto

%Configuraciones del documento
%\selectlanguage{spanish}		%Elijo idioma español

%Tweaks
%\setlength{\parindent}{0mm}		%Sangría de 1a. línea
%\setlength{\hoffset}{-5.4mm}		%
%\setlength{\voffset}{-5.4mm}		%
%\setlength{\topmargin}{0mm}		%
%\setlength{\oddsidemargin}{5mm}	%
%\setlength{\evensidemargin}{5mm}	%
%\setlength{\marginparsep}{5mm}	%
%\setlength{\headheight}{12.5mm}	%
%\setlength{\headsep}{2.5mm}		%
%\setlength{\footskip}{10mm}		%
%\setlength{\textwidth}{15cm}		%
%\setlength{\textheight}{232mm}	%
%\setlength{\fboxrule}{.1pt}
%\setlength{\parskip}{.5\baselineskip}
%Colores
\definecolor{negro}	{cmyk}{0,0,0,1}
\definecolor{marron}	{cmyk}{0,.5,1,.41}
\definecolor{rojo}	{cmyk}{0,1,1,0}
\definecolor{naranja}	{cmyk}{0,.35,1,0}
\definecolor{amarillo}	{cmyk}{0,0,1,0}
\definecolor{verde}	{cmyk}{1,0,1,0}
\definecolor{azul}	{cmyk}{1,1,0,0}
\definecolor{violeta}	{cmyk}{.45,1,0,0}
\definecolor{gris}	{cmyk}{0,0,0,.5}
\definecolor{blanco}	{cmyk}{0,0,0,0}
\definecolor{dorado}	{cmyk}{0,.16,1,0}
\definecolor{plateado}	{cmyk}{0,0,0,.25}

%Comandos personalizados
\newcommand{\T}{\textrm}%Para escribir texto común cuando en modo math

% Remarca un ``texto insertado'' en rojo. Necesita el backage color.
\newenvironment{ins}{\color{red}$>$}{$<$}
\newcommand{\iins}[1]{\color{red}$>$#1$<$}

% Tacha un texto. Depende del package ulem.
\newcommand{\tachar}{\sout}

%\newcommand{\}

%\begin{pspicture}
%\def\tierra(#1){%Para dibujar el símbolo de tierra en el entorno PSTricks
%	\rput(#1){
%		\psdot(0,0)
%		\psline(0,0)(0,-0.45)
%		\psline(-0.5,-0.45)(0.5,-0.45)
%		\psline(-0.35,-0.6)(0.35,-0.6)
%		\psline(-0.2,-0.75)(0.2,-0.75)
%	}%
%}
%\end{pspicture}

%\newcommand{\codigo}[2]{%Para generar un recuadro con código
%	%\setlength{\hrulewidth}{0.1pt}
%	\begin{flushleft}
%	\underline{#1}
%	\begin{tabular}{@{\quad}|l}
%		\begin{minipage}{.85\textwidth}\smallskip{#2}
%	\end{minipage}\end{tabular}\end{flushleft}%
%}

%\newcommand{\filecodigo}[1]{%Insertar código verbatim desde un archivo
%\codigo{#1}{\verbatiminput{#1}}}%Requiere el paquete verbatim
%\newcommand{\filecodigobis}[1]{{\verbatiminput{#1}}}%Requiere el paquete verbatim

%\newcommand{\grafico}[3][1]{%Para generar un plot de un archivo con coords.
%%\def\deequis=#1
%\begin{minipage}{0.5\textwidth}\begin{center}
%\begin{pspicture}(6,5)
%	\psgrid[subgriddiv=1,gridlabels=0pt,gridwidth=.1pt](1,3)(1,1)(6,5)
%	\psset{xunit=5cm,yunit=2cm}
%	\fileplot[linewidth=1pt,linecolor=blue,origin={0.2,1.5}]{#2}
%	\psset{xunit=1cm,yunit=1cm}
%	\psaxes[Dx=#1,dx=5,Oy=-1,Dy=1,dy=2]{-}(0.9,1)(6,5)
%	\rput(4,0.4){\textsl{#3}}
%\end{pspicture}\end{center}\end{minipage}}

%\newcommand{\aclaracion}[1]{%Dibuja un recuadrito aclaratorio
%\smallpencil\ \begin{minipage}{0.9\textwidth}
%\vspace*{6pt}{#1}\smallskip\end{minipage}}

%\newcommand{\consigna}[1]{%Consigna - Slanted
%\leftpointright\ \parbox[t]{0.9\textwidth}{\textsl{#1}\vspace{8pt}}}

%\newcommand{\pinterno}[2]{%Consigna - Slanted
%\left\langle #1 , #2 \right\rangle}

%\newcommand{\eqncode}[2]{%
%\begin{center}
%\begin{tabular}{l@{\hspace{0.5cm}}r}
%\begin{minipage}{.4\textwidth}
%\begin{equation*}
%#1
%\end{equation*}
%\end{minipage}
%&
%\fbox{\begin{minipage}{.4\textwidth}
%%\setlength{\parskip}{4mm}
%\filecodigobis{#2}
%\end{minipage}}
%\end{tabular}
%\end{center}
%}

%\newcommand{\eqncodeb}[2]{%
%\begin{center}\begin{tabular}{l@{\hspace{0.5cm}}r}
%\begin{minipage}{.4\textwidth}#1\end{minipage} &
%\fbox{\begin{minipage}{.4\textwidth}\filecodigobis{#2}\end{minipage}}
%\end{tabular}\end{center}}

%\newenvironment{matemcode}[1]{\newline
%\begin{tabular}{l@{\hspace{0.5cm}}r}
%\begin{minipage}{.4\textwidth}
%\parbox[t]{.4\textwidth}{\begin{equation*}#1\end{equation*}}\end{minipage}
%&\begin{Sbox}\begin{minipage}{.4\textwidth}}
%{\end{minipage}\end{Sbox}\fbox{\TheSbox}\end{tabular}\newline}

%\newenvironment{encuadrar}[1]{\begin{Sbox}\begin{varwidth}{#1\textwidth}}
%{\end{varwidth}\end{Sbox}\fbox{\TheSbox}}

%\newenvironment{enunciado}
%{\leftpointright\ \begin{varwidth}[t]{0.9\textwidth}\textsl}
%{\end{varwidth}\vspace{8pt}}

%\newenvironment{parboxenv}{\begin{Sbox}}
%{\end{Sbox}\parbox[t]{.9\textwidth}{\TheSbox}}

%\newenvironment{pvi}{\begin{equation*}\begin{cases}}
%{\end{cases}\end{equation*}}

% Escribe el texto que le paso como parámetro con letra de ancho fijo.
%\newcommand{\mono}[1]{{\tt #1}}

%\title{\titulo}
%\author{\autor}
%\date{\fecha}

%% si uso pdflatex, me setea las propiedades del pdf de salida
%\ifpdf\pdfinfo{/Title    (\tituloPDF)
%               /Author   (\autorPDF)
%               /Subject  (\asuntoPDF)
%               /Keywords (\clavesPDF)}\fi
