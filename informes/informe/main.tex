%\documentclass[conference,spanish,a4paper,10pt,oneside,final]{IEEEtran}
\nonstopmode
\documentclass[conference,spanish,a4paper,10pt,oneside,final]{tfmpd}


% esto me setea la variable pdf dependiendo del valor de \pdfoutput, que es >0
% sólo cuando estoy usando pdflatex para compilar el documento
\newif\ifpdf
\ifnum\pdfoutput<0
\pdffalse\fi
\ifnum\pdfoutput=0
\pdffalse\fi
\ifnum\pdfoutput>0
\pdftrue\fi

%\makeatletter
%\def\markboth#1#2{\def\leftmark{\@IEEEcompsoconly{\sffamily}\MakeUppercase{\protect#1}}%
%\def\rightmark{\@IEEEcompsoconly{\sffamily}\MakeUppercase{\protect#2}}}
%\makeatother


%
% ===
% === I18n / L10n
% ===
%
% babel me da separación de sílabas para palabras en el idioma que le paso como
%       argumento opcional. <-- Éste hay que pasarlo en \documentclass
%\usepackage{babel}
%
% inputenc define la codificación de caracteres del código fuente, acá utf8.
\usepackage[utf8]{inputenc}
%
% ===
% === Gráficos
% ===
% 
% pst-pdf me permite usar PSTricks con pdflatex. Necesito cargarlo sólo si está
%         definida la variable pdf, por eso está entre \ifpdf ... \fi
\ifpdf\usepackage{pst-pdf}\fi
%
% color me permite usar colores en el documento.
\usepackage{color}
%
% graphicx me da el comando \includegraphics para insertar imágenes (?)
\usepackage{graphicx}
%
% pstricks es un conjunto de macros basadas en PostScript para TeX, en
%          castellano, me da un entorno pstricks y comandos que uso dentro de
%          éste, que me sirven para dibujar figuras/diagramas/etc de manera
%          relativamente simple.
%\usepackage{pstricks}
%
% pst-circ me da macros para pstricks que me dibujan elementos de circuitos
%\usepackage{pst-circ}
%
% 
%\usepackage{pst-plot}		%Para dibujar una curva a partir de un archivo
%\usepackage{pst-2dplot}		%Para plotear. entorno pstaxes
%
% ===
% === Verbatims
% ===
%
% verbatim es una reimplementación de los entornos verbatim[*]
%          provee el comando \verbatiminput{archivo} y el entorno comment, que
%          hace que LaTeX ignore directamente todo lo que está adentro
%\usepackage{verbatim}
%
% moreverb implementa el entorno verbatimtab indentando los tabs que encuentre,
%          y también el entorno listing, que pone números de línea al verbatim.
%          Para cambiar el ancho de la tabulacion, uso
%          \renewcommand\verbatimtabsize{<ancho del tab>\relax}
%          También define el entorno boxedverbatim.
%\usepackage{moreverb}
%
% listings me da el entorno lstlisting con resaltado de sintaxis.
%          Para setear el lenguaje del código, hago \lstset{language=<lang>}
%\usepackage{listings}
%
% url es un verbatim para escribir URL's que permite linebreaks dentro de ésta.
%     para usarlo, \url{<URL>}
\usepackage{url}
%
%
%
%
%
%\usepackage{mdwlist}		%Para listas mas compactas
%\usepackage{textcomp}		%Para algunos símbolos
%\usepackage{colortbl}		%Para celdas de colores en tablas
%\usepackage{fancyhdr}		%Para encabezados/pie
%\usepackage{bbold}		%Fuente bb para modo math: \mathbb{R} = reales
%\usepackage{dsfont}		%Fuente ds para modo math: \mathds{R} = reales
\usepackage{multirow}		%Para "combinar" celdas en tablas
%\usepackage{float}		%Para cuadros, figuras, etc copadas
%\usepackage{fancybox}		%Para recuardos de texto con bordes "fancy"
%\usepackage{dingbat}		%Para dingbats
%%\usepackage{marginal}		%Para notas al margen que no puedo hacer andar
\usepackage{amsmath}		%Para enornos matemáticos mas flexibles
%\usepackage{varwidth}		%varwidth es un minipage que se ajusta al ancho mínimo
\usepackage{pslatex}            % setea fuentes Times, Helvetica y Courier ``angosta''

\usepackage[normalem]{ulem}     %para under/overlining. normalem hace que \emph se comporte como siempre

%
% Propiedades del documento: título, autor, etc
%
%% \newcommand{\titulo}{{\large FICH --- UNL}\\Inteligencia Computacional
%% 2009\\Trabajo Final}
%% \newcommand{\autor}{Abratte, Pablo \and Torrez, Mauro}
%% \newcommand{\fecha}{\today}
%% \newcommand{\tituloPDF}{Trabajo Final IC 2009}
%% \newcommand{\autorPDF}{Abratte, Torrez}
%% \newcommand{\asuntoPDF}{}
%% \newcommand{\clavesPDF}{}
%

%Estilos de texto
\newcommand{\resalt}{\colorbox{green}}	%Resaltado - Fondo verde
\newcommand{\sfbf}{\bfseries\textsf}	%Slanted + Bold
\newcommand{\eng}{\textit}			%Palabra en inglés - Itálica
\newcommand{\mean}{\textsl}			%Significado de una sigla - Slanted
\newcommand{\defin}{\textbf}			%Definición - Negrita
\newcommand{\R}{\mathds}			%Para escribir R de Reales, N de Nat
\newcommand{\N}{\mathbf}			%Para escribir R de Reales, N de Nat
\newcommand{\lil}[1]{\footnotesize #1}
\newcommand{\mono}[1]{{\tt #1}}         % Monoespaciado


%Símbolos
\newcommand{\y}{\wedge}			%Y (Lógica)
\newcommand{\ve}{\vee}			%O (Lógica)
\newcommand{\ent}{\supset}		%Entonces (Lógica)
\newcommand{\dimp}{\leftrightarrow}	%Doble implicativo, equivalencia (Lógica)
\newcommand{\sii}{\leftrightarrow}	%Si y sólo si (Lógica)
\newcommand{\equi}{\equiv}		%Equivalencia (Lógica)
\newcommand{\portanto}{\vdash}	%Por lo tanto (Lógica)
\newcommand{\por}{\cdot}		%Producto punto

%Configuraciones del documento
%\selectlanguage{spanish}		%Elijo idioma español

%Tweaks
%\setlength{\parindent}{0mm}		%Sangría de 1a. línea
%\setlength{\hoffset}{-5.4mm}		%
%\setlength{\voffset}{-5.4mm}		%
%\setlength{\topmargin}{0mm}		%
%\setlength{\oddsidemargin}{5mm}	%
%\setlength{\evensidemargin}{5mm}	%
%\setlength{\marginparsep}{5mm}	%
%\setlength{\headheight}{12.5mm}	%
%\setlength{\headsep}{2.5mm}		%
%\setlength{\footskip}{10mm}		%
%\setlength{\textwidth}{15cm}		%
%\setlength{\textheight}{232mm}	%
%\setlength{\fboxrule}{.1pt}
%\setlength{\parskip}{.5\baselineskip}
%Colores
\definecolor{negro}	{cmyk}{0,0,0,1}
\definecolor{marron}	{cmyk}{0,.5,1,.41}
\definecolor{rojo}	{cmyk}{0,1,1,0}
\definecolor{naranja}	{cmyk}{0,.35,1,0}
\definecolor{amarillo}	{cmyk}{0,0,1,0}
\definecolor{verde}	{cmyk}{1,0,1,0}
\definecolor{azul}	{cmyk}{1,1,0,0}
\definecolor{violeta}	{cmyk}{.45,1,0,0}
\definecolor{gris}	{cmyk}{0,0,0,.5}
\definecolor{blanco}	{cmyk}{0,0,0,0}
\definecolor{dorado}	{cmyk}{0,.16,1,0}
\definecolor{plateado}	{cmyk}{0,0,0,.25}

%Comandos personalizados
\newcommand{\T}{\textrm}%Para escribir texto común cuando en modo math

% Remarca un ``texto insertado'' en rojo. Necesita el backage color.
\newenvironment{ins}{\color{red}$>$}{$<$}
\newcommand{\iins}[1]{{\color{red}$>$#1$<$}}

% Tacha un texto. Depende del package ulem.
\newcommand{\tachar}{\sout}

% Tacha doble
\newcommand{\Tachar}[1]{\xout{\xout{#1}}}

%\newcommand{\}

%\begin{pspicture}
%\def\tierra(#1){%Para dibujar el símbolo de tierra en el entorno PSTricks
%	\rput(#1){
%		\psdot(0,0)
%		\psline(0,0)(0,-0.45)
%		\psline(-0.5,-0.45)(0.5,-0.45)
%		\psline(-0.35,-0.6)(0.35,-0.6)
%		\psline(-0.2,-0.75)(0.2,-0.75)
%	}%
%}
%\end{pspicture}

%\newcommand{\codigo}[2]{%Para generar un recuadro con código
%	%\setlength{\hrulewidth}{0.1pt}
%	\begin{flushleft}
%	\underline{#1}
%	\begin{tabular}{@{\quad}|l}
%		\begin{minipage}{.85\textwidth}\smallskip{#2}
%	\end{minipage}\end{tabular}\end{flushleft}%
%}

%\newcommand{\filecodigo}[1]{%Insertar código verbatim desde un archivo
%\codigo{#1}{\verbatiminput{#1}}}%Requiere el paquete verbatim
%\newcommand{\filecodigobis}[1]{{\verbatiminput{#1}}}%Requiere el paquete verbatim

%\newcommand{\grafico}[3][1]{%Para generar un plot de un archivo con coords.
%%\def\deequis=#1
%\begin{minipage}{0.5\textwidth}\begin{center}
%\begin{pspicture}(6,5)
%	\psgrid[subgriddiv=1,gridlabels=0pt,gridwidth=.1pt](1,3)(1,1)(6,5)
%	\psset{xunit=5cm,yunit=2cm}
%	\fileplot[linewidth=1pt,linecolor=blue,origin={0.2,1.5}]{#2}
%	\psset{xunit=1cm,yunit=1cm}
%	\psaxes[Dx=#1,dx=5,Oy=-1,Dy=1,dy=2]{-}(0.9,1)(6,5)
%	\rput(4,0.4){\textsl{#3}}
%\end{pspicture}\end{center}\end{minipage}}

%\newcommand{\aclaracion}[1]{%Dibuja un recuadrito aclaratorio
%\smallpencil\ \begin{minipage}{0.9\textwidth}
%\vspace*{6pt}{#1}\smallskip\end{minipage}}

%\newcommand{\consigna}[1]{%Consigna - Slanted
%\leftpointright\ \parbox[t]{0.9\textwidth}{\textsl{#1}\vspace{8pt}}}

%\newcommand{\pinterno}[2]{%Consigna - Slanted
%\left\langle #1 , #2 \right\rangle}

%\newcommand{\eqncode}[2]{%
%\begin{center}
%\begin{tabular}{l@{\hspace{0.5cm}}r}
%\begin{minipage}{.4\textwidth}
%\begin{equation*}
%#1
%\end{equation*}
%\end{minipage}
%&
%\fbox{\begin{minipage}{.4\textwidth}
%%\setlength{\parskip}{4mm}
%\filecodigobis{#2}
%\end{minipage}}
%\end{tabular}
%\end{center}
%}

%\newcommand{\eqncodeb}[2]{%
%\begin{center}\begin{tabular}{l@{\hspace{0.5cm}}r}
%\begin{minipage}{.4\textwidth}#1\end{minipage} &
%\fbox{\begin{minipage}{.4\textwidth}\filecodigobis{#2}\end{minipage}}
%\end{tabular}\end{center}}

%\newenvironment{matemcode}[1]{\newline
%\begin{tabular}{l@{\hspace{0.5cm}}r}
%\begin{minipage}{.4\textwidth}
%\parbox[t]{.4\textwidth}{\begin{equation*}#1\end{equation*}}\end{minipage}
%&\begin{Sbox}\begin{minipage}{.4\textwidth}}
%{\end{minipage}\end{Sbox}\fbox{\TheSbox}\end{tabular}\newline}

%\newenvironment{encuadrar}[1]{\begin{Sbox}\begin{varwidth}{#1\textwidth}}
%{\end{varwidth}\end{Sbox}\fbox{\TheSbox}}

%\newenvironment{enunciado}
%{\leftpointright\ \begin{varwidth}[t]{0.9\textwidth}\textsl}
%{\end{varwidth}\vspace{8pt}}

%\newenvironment{parboxenv}{\begin{Sbox}}
%{\end{Sbox}\parbox[t]{.9\textwidth}{\TheSbox}}

%\newenvironment{pvi}{\begin{equation*}\begin{cases}}
%{\end{cases}\end{equation*}}

% Escribe el texto que le paso como parámetro con letra de ancho fijo.
%\newcommand{\mono}[1]{{\tt #1}}

%\title{\titulo}
%\author{\autor}
%\date{\fecha}

%% si uso pdflatex, me setea las propiedades del pdf de salida
%\ifpdf\pdfinfo{/Title    (\tituloPDF)
%               /Author   (\autorPDF)
%               /Subject  (\asuntoPDF)
%               /Keywords (\clavesPDF)}\fi

%\include{conf/comandos}
%
\begin{document}
\title{Título tentativo del paper (completar!)}
\author{Esteban Fornal, Christian Pfarher, Mauro J. Torrez\\
\textit{Trabajo práctico final de ``Captura y Procesamiento Digital de
Imágenes'', II-FICH-UNL.}}
\markboth{CPDSI: TRABAJO FINAL}{}
\maketitle
%
%
% %%%%%%%%%%%%%%%%%%%%%%%%%%%%%%%%%%%%%%%%%%%%%%%%%%%%%%%%%%%%%%%%%%%%%%%%%%%%%%
%
%
\begin{abstract}
Se describe la implementación de un controlador borroso %(FAM)
que intenta imitar el comportamiento de un jugador humano en un juego de carrera
de autos. Se toman como entradas al controlador una serie de sensores simples
que miden proximidad a otros automóviles y a la banquina. Las salidas del sistema
borroso son la dirección del volante, y la presión aplicada en el acelerador y el
freno.
Se prueba el controlador con dos conjuntos de reglas borrosas y se contrastan con
los resultados obtenidos por jugadores humanos, en distintas situaciones en la
carretera.
\end{abstract}
%
%
% %%%%%%%%%%%%%%%%%%%%%%%%%%%%%%%%%%%%%%%%%%%%%%%%%%%%%%%%%%%%%%%%%%%%%%%%%%%%%%
%
%
\begin{keywords}
Lógica borrosa, Control borroso, FAM--\eng{Fuzzy Associative Memory}, Videojuego, Carrera.
\end{keywords}
%
%
% %%%%%%%%%%%%%%%%%%%%%%%%%%%%%%%%%%%%%%%%%%%%%%%%%%%%%%%%%%%%%%%%%%%%%%%%%%%%%%
%
%
\section{Introducción}
\PARstart{L}{os} videojuegos han evolucionado en los últimos años brindando al jugador una experiencia
que se asemeja cada vez más a la realidad. Las técnicas de Inteligencia Computacional nos permiten
hacer esta experiencia aún más real mediante, entre otras, la posibilidad de modelar el comportamiento
de seres inteligentes.
% ``conciencias'' en f%% orma
%% cuyo comportamiento se parece más bien al de un humano que a un \resalt{aaaaa}

%% La evolución de los videojuegos en los ultimos años ha tendido al enriquecimiento de los modelos para representar fenomenos diversos.
%% La complejidad de los modelos para representar dichos fenomenos hace de
%% las tecnicas de inteligencia computacional herramientas factibles y cada vez mas utilizadas.

%% La actual tendencia de los videojuegos a representar la realidad cada vez con mayor realismo trae aparejada
%% la necesidad de tecnicas que permitan el modelado de comportamientos que se


%%  de las entidades con las que el jugador
%% interactua, buscando que dicho comportamiento se asemeje al de otros jugadores humanos.

%% Motivados por esta situación nos proponemos el desarrollo de un ``controlador inteligente'' para
%% un videojuego rudimentario de carreras de autos que sea capaz de imitar el comportamiento de un jugador humano.  
%meter la cita aca

En este trabajo presentamos un modelo de control borroso de un ``piloto
automático'' para un juego rudimentario de carrera de autos, que a través de una serie de
sensores de proximidad montados en el coche y las reglas codificadas en la FAM,
intentará controlar el automóvil evitando colisiones y salirse de la ruta.

%% De las técnicas posibles para emprender esta tarea, se consideró al control borroso 
%% como la mas apropiada por permitir una representación descriptiva simple y directa del
%% conocimiento empleado por el jugador, además de facilitar una implementación rapida, robusta y eficiente.

Las reglas de la FAM surgen de analizar construcciones lingüísticas que describen
cómo se comportaría un jugador humano al mando del juego.

Como el videojuego utilizado representa una visión simplificada de una situación real, este trabajo 
podría eventualmente utilizarse como base para el desarrollo de un piloto automatico para automóviles.
%
%
% %%%%%%%%%%%%%%%%%%%%%%%%%%%%%%%%%%%%%%%%%%%%%%%%%%%%%%%%%%%%%%%%%%%%%%%%%%%%%%
%
%
\section{Memorias Asociativas Borrosas (FAMs)}
Una FAM es un sistema borroso que codifica \emph{mapeos} o transformaciones entre
conjuntos borrosos. 

El tipo de FAM más simple es aquella que mapea la regla $(A_i,B_i)$, la cual asocia
el conjunto borroso $p-$dimensional $B_i$ con el conjunto $n-$dimensonal $A_i$.
En general, un sistema FAM $F: I^n \rightarrow I^p$ codifica una serie de $m$ reglas
$(A_1,B_1),\ldots,(A_m,B_m)$.

Como cada conjunto representa una cantidad en lenguaje corriente, el conjunto de
reglas en la FAM se define en torno a una asociación lngüística simple, por ejemplo,
$($FRESCO$,$BAJO$)$ podría indicar una relación temperatura $\rightarrow$ nivel del
calefactor.

%\resalt{TODO: indicar cuál es la relación entre $A$ y los $A_i$.}

Dada una entrada real $x$ al sistema, obtendremos una serie de valores de
pertenencia $\mu_{A_{i}}(x), 0\leq\mu_{A_{i}}\leq1$, para cada conjunto de entrada $A_i$, que en la
FAM activa cada regla en diferentes grados;
al evaluar la regla $(A_i,B_i)$, se obtiene $B'_i$,
una versión de $B_i$ parcialmente activado.
Cuanto mayor sea el valor $\mu_{A_{i}}(x)$, más parecido será $B'_i$ a $B_i$.

El conjunto $B$ de salida se obtiene mediante un promedio ponderado de los
conjuntos parcialmente activados $B'_1,B'_2,\ldots,B'_m$:
\begin{equation}
B=w_1B'_1+\cdots+w_mB'_m
\end{equation}
donde $w_i$ es un grado certeza, o frecuencia, de la asociación $(A_i,B_i)$.
En la práctica, se \emph{defuzzifica} comúnmente la salida $B$ a un único
valor real $y$ mediante el método de los centroides, según
\begin{equation}
y=\frac{\sum_{i=1}^{m}{a_ic_i}}{\sum_{i=1}^{m}{a_i}}
\end{equation}
donde $a_i,c_i$ son el área y centroide, respectivamente, del conjunto $B'_i$ parcialmente
activado.

De esta forma podemos obtener para una entrada $x$ la salida $y$ del sistema [1].
%
%
% %%%%%%%%%%%%%%%%%%%%%%%%%%%%%%%%%%%%%%%%%%%%%%%%%%%%%%%%%%%%%%%%%%%%%%%%%%%%%%
%
%
\section{Diseño del controlador}
Partiendo de una implementación base de un juego de carrera de autos, 
%que rebautizamos
%como ``Fuzzy Driver'',
la extendimos codificando una serie de \emph{sensores} montados en
el auto, que devuelven un ``voltaje'' en un rango $\in [0,1]$, los cuales sensan
la distancia a otros autos, en línea recta: emiten una señal de $1V$ cuando
la distancia es máxima, y de $0V$ cuando es mínima (hay algo ``tocando'').
Hemos elegido este rango de valores con el propósito de obtener generalidad.

Los sensores frontales se ubican justo al lado de los faros del auto, de tal
forma que están orientados hacia adelante, y se ubica además un sensor a la 
derecha y otro a la izquierda, apuntando a los laterales.
Los sensores laterales trabajan en un rango de distancias distinto al de los
frontales, y suponemos que son capaces de detectar, además de autos, la
banquina de la pista.

El controlador consiste en dos módulos FAM, uno  comanda la dirección
del volante, y el otro la aceleración y frenado, como se ve en la figura 1.
%
\begin{figure}[tbhp]
%\centerline{\includegraphics[width=7cm]{img/diag-modulos}}
\caption{Diagrama funcional del controlador.}
\label{fig1}
\end{figure}
%
\subsection{Entradas y salidas del sistema}
El módulo de dirección lee las entradas
\mono{dis\-tan\-cia\_fron\-tal\_iz\-quier\-da}, \mono{dis\-tan\-cia\_fron\-tal\_de\-re\-cha},
\mono{dis\-tan\-cia\_la\-te\-ral\_iz\-quier\-da}, y \mono{dis\-tan\-cia\_la\-te\-ral\_de\-re\-cha},
de los respactivos sensores, las fuzzifica, aplica las reglas FAM y
calcula el valor de salida \mono{di\-rec\-ción\_vo\-lan\-te}, que está en el rango
$[-1,1]$. El actuador tornea el volante a la izquierda cuando el valor
es negativo, a la derecha cuando es positivo, lo mantiene centrado cuando
el voltaje de salida es 0.

El módulo de aceleración y frenado tiene las entradas \mono{ve\-lo\-ci\-dad\_ac\-tual},
\mono{dis\-tan\-cia\_fron\-tal\_máxi\-ma}, y \mono{dis\-tan\-cia\_fron\-tal\_mí\-ni\-ma}, donde
estas dos últimas son la máxima y mínima medida en los sensores frontales. De
esta manera se logra simetría en las decisiones en este módulo. Las salidas
son dos, \mono{aceleración} y \mono{frenado}, ambas variando de 0 a 1, donde 0
indica no pisar el pedal, y 1, pisarlo a fondo.

Para obtener las salidas de ambas FAM se utiliza el método de defuzzificación
por centroides, como se describió antes, y los conjuntos de salida son como
se ve en la figura \ref{fig3}. Los conjuntos de fuzzificación para las 
entradas se pueden ver en la figura \ref{fig2}. 
%
\begin{figure}[tbhp]
%\centerline{\includegraphics[width=6cm]{img/conj-entradas}}
\caption{Conjuntos borrosos para fuzzificación de las entradas.}
\label{fig2}
\end{figure}
%
\begin{figure}[tbhp]
%\centerline{\includegraphics[width=6cm]{img/conj-salidas}}
\caption{Conjuntos borrosos para defuzzificación de las salidas.}
\label{fig3}
\end{figure}
%
\subsection{Codificación de las reglas borrosas}
Para obtener las reglas nos hemos basado en proposiciones en lenguaje natural,
como puede ser ``si la distancia medida por los dos sensores frontales es máxima, y
la velocidad actual es media, entonces pisar a fondo el acelerador''. 

La codificación utilizada es la de correlación-mínimo, la regla $(A_1\y A_2\y A_3,B_2)$
se activa una cantidad $a$ dada por
\begin{equation}
a=\T{mín}(a_1(x),a_2(x),a_3(x))
\end{equation}
donde $a_i$ indica el grado de activación del conjunto $A_i$. 
%
%
% %%%%%%%%%%%%%%%%%%%%%%%%%%%%%%%%%%%%%%%%%%%%%%%%%%%%%%%%%%%%%%%%%%%%%%%%%%%%%%
%
%
\section{Evaluación}
La evaluación del controlador borroso se realizó utilizando dos conjuntos de reglas
y comparando su performance con la de jugadores humanos.

En el primer conjunto de reglas borrosas (CR1) utilizado se pretende describir un comportamiento similar al de
de un jugador humano que intenta esquivar a los demás coches en la pista con la menor
disminución posible en la velocidad. 

Mediante modificaciones en dichas reglas se logró un segundo conjunto de
reglas (CR2) que modela el comportamiento de un conductor conservador que intenta
esquivar al resto de los coches evitando velocidades altas y movimientos pronunciados. 

Los parámetros evaluados fueron: la cantidad de colisiones, la distancia total recorrida en píxeles 
y el tiempo fuera de pista.
Se realizaron simulaciones de 3 minutos cada una utilizando una combinación de 5 y 10 competidores
simultáneos (máximo) manejados por un controlador difuso (CD) igual al del coche principal, y también otros con movimiento 
vertical uniforme (MVU) a su velocidad máxima. En todos los casos la velocidad máxima
de los competidores se eligió de forma aleatoria
entre 300 y 700 píxeles por segundo, mientras que la del coche principal fue fija
y de 500 píxeles por segundo.

%Para evaluar la distancia total recorrida se utilizo el producto de la velocidad instantánea por el diferencial de tiempo
%transcurrido en intervalos pequeños.

%original  Se consideró al coche principal como fuera de la pista cuando un porcentaje del mismo mayor al 30\%
% se encontraba fuera de la misma.
%maurete modif
Se consideró al coche principal como fuera de la pista cuando más del 30\% de éste se encontraba fuera de la misma.
%
Para cada una de las pruebas de los controladores se realizaron 3 simulaciones y se promediaron sus resultados.
Para obtener medidas de la performance de jugadores humanos (JH) se promediaron los tiempos
de dos jugadores.

\begin{figure}[tbhp]
%\centerline{\includegraphics[width=5cm]{img/evaluacion}}
\caption{Resultados de la evaluacion del controlador borroso para (a) cantidad de colisiones,
 (b) distancia recorrida,  (c) tiempo fuera de la pista.}
\label{evaluacion}
\end{figure}

En la figura \ref{evaluacion}(a) pueden observarse los resultados obtenidos respecto a la cantidad de colisiones. 
En todos los casos, éstas aumentaron con un mayor número de competidores. En general, se registraron menos colisiones con los
competidores MVU. Con 10 competidores el CR1 mostró una cantidad de colisiones levemente menor que los JH.
En todos los casos el CR2 logró en general una mejor performance.

La distancia recorrida total (figura \ref{evaluacion} (b)) fue, en todos los casos, mayor con los JH.
El CR2 registró las menores distancias, y en la totalidad de los casos se registraron mayores distancias con competidores
MVU y con un menor número de competidores.

Al medir el tiempo fuera de pista (figura \ref{evaluacion} (c)),
con 10 competidores, CR2 y JH registraron tiempos similares. En todos los
casos el CR1 registró los mayores tiempos fuera de pista. Con 5 competidores los tiempos registrados fueron
menores que con 10 competidores, y las diferencias más marcados, siendo JH el que menor tiempo fuera de pista registró
y el conjunto de reglas 2 con tiempos un poco mayores.

La implementación rudimentaria de los sensores resulta el principal cuello de botella a la hora de obtener
entradas de calidad para la FAM, y trunca las posibilidades a la hora de plantear las reglas del controlador borroso.
El jugador humano es capaz de percibir dichas entradas y además utilizarla para realizar procesos
de razonamiento muy complejos y difíciles de modelar, que le permiten obtener respuestas mucho mejor ajustadas
al resultado que se espera lograr.

Además, existen otros factores como el juicio de valor que puede hacerse
respecto a una situación determinada. Por ejemplo si se trata de rebasar
un auto pero el único lugar es saliendo de la pista, puede elegir
rebasarlo por la banquina para no perder velocidad, o no esquivarlo y frenar para no chocarlo.
A pesar de ello, la performance obtenida por el controlador es equiparable en varios aspectos a la de un jugador humano
medianamente experimentado. 

Finalmente, en la figura \ref{evaluacion}(a) se observa un bajo número de colisiones para el CR2,
excepto en el caso de 5 competidores MVU donde dicho valor es drásticamente alto.
Esto puede ser atribuido una situación puntual que representa una limitación inherente a la implementación de los sensores.
Dicha situación ocurre cuando se intenta rebasar un auto cercano a la banquina por la misma dirección de esta última.
Las reglas que dicen que el coche debe esquivar y que no debe salir de la pista se contradicen y éste nunca puede rebasar
al competidor y continúa chocándolo repetidamente desde atrás en cada intento.
%
%
% %%%%%%%%%%%%%%%%%%%%%%%%%%%%%%%%%%%%%%%%%%%%%%%%%%%%%%%%%%%%%%%%%%%%%%%%%%%%%%
%
%
\section{Conclusiones}
Hemos presentado un sistema de control borroso para un piloto automático en un videojuego
de carrera de autos sencillo. El uso de FAMs permite la especificación de las reglas
en forma directa mediante proposiciones lingüístcas. 
El rendimiento del controlador fue bueno, considerando la sencillez de especificación
de las reglas y la muy baja complejidad de los sensores, los resultados han sido comparables
a los de un jugador humano.

Este modelo de controlador puede ser fácilmente implementable para manejar los competidores
en un juego de carreras más complejo, dotando a éste de mayor parecido con la realidad, ya
que simula bastante bien conductores humanos. Se deberá considerar en este caso la implementación
de, al menos, un sensor de velocidad lateral y mejorar la disposición y número de sensores frontales.

Los sensores utilizados en la
implementación no difieren en mucho de detectores reales disponibles en el mercado,
sin embargo, se deberán agregar sensores más complejos, muy posiblemente algún clasificador
mediante una red neuronal y elaborar reglas con criterios
minuciosamente verificados antes de llevar un controlador de este tipo a una implementación
en un automóvil real, ya que evitar colisiones o salirse de pista no es una opción en el
mundo real.
%
%
% %%%%%%%%%%%%%%%%%%%%%%%%%%%%%%%%%%%%%%%%%%%%%%%%%%%%%%%%%%%%%%%%%%%%%%%%%%%%%%
%
%
\section{Trabajos futuros}
A partir del diseño aquí presentado, seguiremos investigando esta técnica con las siguientes
posibilidades:
\begin{itemize}
\item Agregado de más sensores, en configuraciones diferentes, uno de los que sospechamos
      mejoraría mucho el rendimiento es uno de velocidad/aceleración lateral.
\item Utilización de sensores más ``inteligentes'', capaces de reconocer vehículos de
      diferentes tipos, mediante un clasificador de patrones neuronal.
\item Posibilidad de modelar una carretera donde solamente circulen autos manejados por
      controladores borrosos, para estudiar el comportamiento de los vehículos,
      y entender cuáles son las limitaciones que hacen que estas técnicas no se apliquen en
      la realidad.
\item Aplicar técnicas para adaptabilidad en los rangos de los conjuntos de entrada/salida. 
\item Implementación del mismo controlador mediante una red neuronal para comparar el rendimiento
      de ambos.
\end{itemize}

\nocite{*}
\bibliographystyle{tfmpd}
\bibliography{tfmpd}
\end{document}

